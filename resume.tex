%%%%%%%%%%%%%%%%%%%%%%%%%%%%%%%%%%%%%%%
% Deedy - One Page Two Column Resume
% LaTeX Template
% Version 1.2 (16/9/2014)
%
% Original author:
% Debarghya Das (http://debarghyadas.com)
%
% Original repository:
% https://github.com/deedydas/Deedy-Resume
%
% IMPORTANT: THIS TEMPLATE NEEDS TO BE COMPILED WITH XeLaTeX
%
% This template uses several fonts not included with Windows/Linux by
% default. If you get compilation errors saying a font is missing, find the line
% on which the font is used and either change it to a font included with your
% operating system or comment the line out to use the default font.
% 
%%%%%%%%%%%%%%%%%%%%%%%%%%%%%%%%%%%%%%
% 
% TODO:
% 1. Integrate biber/bibtex for article citation under publications.
% 2. Figure out a smoother way for the document to flow onto the next page.
% 3. Add styling information for a "Projects/Hacks" section.
% 4. Add location/address information
% 5. Merge OpenFont and MacFonts as a single sty with options.
% 
%%%%%%%%%%%%%%%%%%%%%%%%%%%%%%%%%%%%%%
%
% CHANGELOG:
% v1.1:
% 1. Fixed several compilation bugs with \renewcommand
% 2. Got Open-source fonts (Windows/Linux support)
% 3. Added Last Updated
% 4. Move Title styling into .sty
% 5. Commented .sty file.
%
%%%%%%%%%%%%%%%%%%%%%%%%%%%%%%%%%%%%%%%
%
% Known Issues:
% 1. Overflows onto second page if any column's contents are more than the
% vertical limit
% 2. Hacky space on the first bullet point on the second column.
%
%%%%%%%%%%%%%%%%%%%%%%%%%%%%%%%%%%%%%%

\documentclass[]{deedy-resume-openfont}
\documentclass[10pt]{report}
\usepackage{fancyhdr}
\usepackage{xcolor}
\fontspec{[fontawesome-webfont.ttf]}
\usepackage{fontawesome}
\pagestyle{fancy}
\fancyhf{}
 
\begin{document}


%%%%%%%%%%%%%%%%%%%%%%%%%%%%%%%%%%%%%%
%
%     TITLE NAME
%
%%%%%%%%%%%%%%%%%%%%%%%%%%%%%%%%%%%%%%
\namesection{}{Aryaman Jain}{
Sophomore @ Computer Science and Enginnering, IIT Kharagpur}
%%%%%%%%%%%%%%%%%%%%%%%%%%%%%%%%%%%%%%
%
%     COLUMN ONE
%
%%%%%%%%%%%%%%%%%%%%%%%%%%%%%%%%%%%%%%

\begin{minipage}[t]{0.33\textwidth} 

%%%%%%%%%%%%%%%%%%%%%%%%%%%%%%%%%%%%%%
%     EDUCATION
%%%%%%%%%%%%%%%%%%%%%%%%%%%%%%%%%%%%%%

\section{Education} 

\subsection{Indian Institute of \\ Technology Kharagpur}
\descript{Dual Degree in Computer \\ Science and Engineering \\ CGPA: 9.17/10}
\location{2018-2023 | West Bengal, India}
\sectionsep

%%%%%%%%%%%%%%%%%%%%%%%%%%%%%%%%%%%%%%
%     LINKS
%%%%%%%%%%%%%%%%%%%%%%%%%%%%%%%%%%%%%%

\section{INTEREST AREAS}
I love to learn new things and it is one of the reasons to choose Computer Science as it never fails to surprise us with new things.

I started doing machine learning after trying different areas of software engineering and instantly fell in love with it.
I am mainly interested in applications of machine learning and deep learning in real life.

Other than machine learning I also know web development.

\sectionsep

%%%%%%%%%%%%%%%%%%%%%%%%%%%%%%%%%%%%%%
%     CONTACT
%%%%%%%%%%%%%%%%%%%%%%%%%%%%%%%%%%%%%%

\section{CONTACT} 
\descript{LLR Hall, \\ IIT Kharagpur,\\ West Bengal - 721302 \\ Phone: +91-8223801399}
\sectionsep
%%%%%%%%%%%%%%%%%%%%%%%%%%%%%%%%%%%%%%
%     LINKS
%%%%%%%%%%%%%%%%%%%%%%%%%%%%%%%%%%%%%%

\section{LINKS} 
\faEnvelope\ {\href{mailto:mukul.csiitkgp@gmail.com}{ jainaryaman123@gmail.com}}\\
\faGithub\ {\href{https://github.com/Aryamn}{Aryamn}}\\
\faLinkedinSquare\ {\href{https://www.linkedin.com/in/aryaman-jain-77784817a/}{aryaman-jain}}\\

\sectionsep

%%%%%%%%%%%%%%%%%%%%%%%%%%%%%%%%%%%%%%
%     COURSEWORK
%%%%%%%%%%%%%%%%%%%%%%%%%%%%%%%%%%%%%%

\section{Coursework}
\emph{T - Theory | L - Laboratory}\\
\descript{Completed}
Programming \& Data Structures (T/L)\\
Discrete Structures (T)\\
Algorithms-1 (T/L) \\
\descript{Ongoing}
Software Engineering (T/L) \\
Formal Language \& Automata Theory (T) \\
Switching Circuits \& Logic Design (T/L) \\ 
Probablity and Statistics (T)\\
\descript{Online Courses}
% href{https://www.coursera.org/learn/machine-learning}{Machine Learning - Andrew Ng}\\
\href{https://www.coursera.org/specializations/deep-learning}{Deep Learning - Andrew Ng}\\
% \href{https://www.coursera.org/learn/deep-neural-network?specialization=deep-learning}{DL - Improving Deep Neural Networks}\\
 \href{https://www.coursera.org/learn/convolutional-neural-networks?action=enroll&authMode=signup&specialization=deep-learning}{Convolutional Neural Networks}\\
\href{https://www.coursera.org/learn/convolutional-neural-networks?action=enroll&authMode=signup&specialization=deep-learning}{Deep Neural Networks with PyTorch}\\

\sectionsep
%%%%%%%%%%%%%%%%%%%%%%%%%%%%%%%%%%%%%%
%
%     COLUMN TWO
%
%%%%%%%%%%%%%%%%%%%%%%%%%%%%%%%%%%%%%%

\end{minipage} 
\hfill
\begin{minipage}[t]{0.66\textwidth} 


%%%%%%%%%%%%%%%%%%%%%%%%%%%%%%%%%%%%%%
%     PROJECTS
%%%%%%%%%%%%%%%%%%%%%%%%%%%%%%%%%%%%%%

\section{PROJECTS}

\runsubsection{Blog Web App}
\vspace{\topsep} % Hacky fix for awkward extra vertical space
\begin{tightemize}
\item Made a Web App with flask based back-end and HTML,CSS,JavaScript based front-end. Used SQLAlchemy for storing database.
\end{tightemize}

\sectionsep

\runsubsection{{Image Classifier}}
\descript{Convolutional Neural Net}


\begin{tightemize}
\item Made a image classifier on CIFAR10 using deep learning.  
\item Used Convolutional Neural Net architecture with the help of PyTorch library .
\end{tightemize}
\sectionsep

\runsubsection{Path Planning}
\descript{AGV Task Round}
\begin{tightemize}
\item Optimized multiple data structures and path finding algorithm(Dijkstra) to find a solution for a given maze, with multiple constraints towards emulating a real car. Implemented in p5.js.

\end{tightemize}
\sectionsep

\runsubsection{General Championship Data Analytics}
\descript{Inter-hall}
\begin{tightemize}
\item Implemented ARIMA Model on the given time series data to predict future results.
\item Gave important inferences about the data.

\end{tightemize}
\sectionsep

%%%%%%%%%%%%%%%%%%%%%%%%%%%%%%%%%%%%%%
%   SKILLS
%%%%%%%%%%%%%%%%%%%%%%%%%%%%%%%%%%%%%%

\section{Technical Skills}
\begin{tabular}{r|p{15cm}}
\textsc{Programming Languages} & \textbf{ C | C++ | Python | JavaScript | GNU Octave (Familiar)} \\
\textsc{Libraries / Frameworks} & \textbf{Pytorch | Pandas | Flask | OpenCV | Numpy | sklearn |  |  }\\
\textsc{Databases} & \textbf{SQLAlchemy}\\
\textsc{Systems / Platforms} & \textbf{Git | Linux}\\ 
\textsc{Web Technolgies} & \textbf{HTML+CSS	| BeautifulSoup} \\
\end{tabular}
%----------------------------------------------------------------------------------------
%	Scholastic Achievements
%----------------------------------------------------------------------------------------

\section{ACHIEVEMENTS}

\vspace{\topsep} % Hacky fix for awkward extra vertical space
\begin{tightemize}

\item{Codechef :} 4 star rating competitive coder on Codechef platform.
\item{Facebook Hacker Cup:} Participated and cracked first round in Facebook Hacker Cup 2019.
\item{Google Code Jam :} Participated and cracked first round in Google Code Jam 2019.
\item{JEE Advanced \emph{AIR 376}:} Under top 0.16\% amongst more than 2,00,000 students \\
\item {JEE Main \emph{AIR 20001}:} In top 0.15\% amongst more than 12,00,000 students. \\
\end{tightemize}
\sectionsep

%	Activities and Leadership
%----------------------------------------------------------------------------------------

\section{Activities \& Leadership}

\vspace{\topsep} % Hacky fix for awkward extra vertical space
\begin{tightemize}


\item {\emph{Core Team Member, ProDex}:} Society of people having enthusiasm towards Product Designing.\\
\item{\emph{Product Designing Involvement:}} Part of Product Design team of my hall.
\item{\emph{Data Analytics:}} Part of Data Analytics team of my hall.
\item{\emph{Mentor:}} Mentored a team of 4 students regarding Product Design competition.
\item{\emph{Secretary:}} Managed several hall events related to technology in my hall.

\end{tightemize}
\sectionsep

\end{minipage} 
\end{document}  \documentclass[]{article}\